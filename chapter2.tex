\chapter{نحوه‌ی نگارش قسمت‌های مختلف پایان‌نامه}

در این سبک متن، فایل‌های مختلفی وجود دارد که 
در هرکدام قسمت خاصی از صفحات ابتدایی و همچنین متن اصلی پایان‌نامه قرار دارد.
در این فصل این قسمت‌ها به ترتیب معرفی شده و فایل‌های متناسب با آنها نیز معرفی شده‌اند.

\قسمت{مشخصات پایان‌نامه}
منظور از مشخصات پایان‌نامه عبارت است از:
\شروع{شمارش}
\فقره عنوان پایان‌نامه
\فقره نام دانشجو یا دانشجویان و شماره‌ی دانشجویی ‌آنها
\فقره نام استاد (اساتید) راهنما
\فقره نام استاد (اساتید)  مشاور
\فقره استاد (اساتید) ممتحن داخلی
\فقره استاد (اساتید) ممتحن خارجی
\فقره نماینده‌ی تحصیلات تکمیلی (برای مقطع ارشد و دکتری)
\پایان{شمارش}

این اطلاعات به‌صورت فارسی و برخی از آنها (موارد 1 تا 4) به‌صورت انگلیسی باید وارد شوند.
برای وارد کردن این اطلاعات، از فایل‌های \lr{fa\_title}  و \lr{en\_title}  استفاده می‌شود.

\قسمت{\فا{تشکر و قدردانی}}
چنانچه به تشکر و قدردانی نیاز باشد، از فایل \lr{Acknowledge.tex } استفاده می‌شود.
این فایل به‌صورت اختیاری بوده و در صورت علاقه باید آن‌را پر کرد.
برای آنکه تعداد فایل‌ها در پوشه‌ی اصلی زیادنشود، این فایل  درون یک پوشه‌ی مجزا به 
نام \lr{FrontMat}‌ قرار دارد.


\قسمت{\فا{نیایش}}
چنانچه به ذکر نیایش نیاز باشد، از فایل \lr{Invocation.tex  } استفاده می‌شود.
این فایل به‌صورت اختیاری بوده و در صورت علاقه باید آن‌را پر کرد.
برای آنکه تعداد فایل‌ها در پوشه‌ی اصلی زیادنشود، این فایل  درون یک پوشه‌ی مجزا به 
نام \lr{FrontMat}‌ قرار دارد.

\قسمت{\فا{تقدیم}}
صفحه‌ی تقدیم نیز یک صفحه‌ی اختیاری بوده و متن آن در فایل  \lr{Dedicate.tex } 
نوشته می‌شود.
 این فایل نیز  درون پوشه‌ی  \lr{FrontMat}‌ قرار دارد.


\قسمت{\فا{فهرست نمادها}}
برای نگارش فهرست نمادها، فایلی به نام   \lr{Symbols.tex } 
 درون پوشه‌ی  \lr{FrontMat}‌ قرار دارد. در این فایل یک نمونه‌ی کوچک نیز قرار داده شده
تا طبق آن نگارش انجام شود. برای نوشتن این علائم، یک دستور به نام \lr{namad}  تعریف
شده که فرمت آن به صورت زیر است:
\[
\backslash \text{namad}\{Symb\}\{Description\}\{Dimension\}
\]
همان‌گونه که از این دستور مشخص است، اولین آرگومان خود 
نماد بوده، دومین آرگومان تعریف آن و آرگومان سوم واحد آنرا مشخص می‌کند.




\قسمت{\فا{مراجع}}
فایل  \lr{References.tex} برای ذخیره‌کردن مراجع به‌کار می‌رود. 



\قسمت{\فا{فصل‌های مختلف پایان‌نامه}}
برای نگارش فصل‌های مختلف پایان‌نامه، کافی است مانند  فایل‌هایی که فصل 1 و 2
در آن نوشته شده، به تعداد مورد نیاز کپی کرده و در هر فایل، یک فصل از پایان‌نامه
ذخیره شود.








