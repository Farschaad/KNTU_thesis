% در این فایل، عنوان پایان‌نامه، مشخصات خود، متن تقدیمی‌، ستایش، سپاس‌گزاری و چکیده پایان‌نامه را به فارسی، وارد کنید.
% توجه داشته باشید که جدول حاوی مشخصات پروژه/پایان‌نامه/رساله و همچنین، مشخصات داخل آن، به طور خودکار، درج می‌شود.
%%%%%%%%%%%%%%%%%%%%%%%%%%%%%%%%%%%%
% دانشگاه خود را وارد کنید
\university{دانشگاه صنعتی خواجه نصیرالدین طوسی}
% دانشکده، آموزشکده و یا پژوهشکده  خود را وارد کنید
\faculty{دانشکده مهندسی مکانیک}
% گروه آموزشی خود را وارد کنید
\department{گروه سیستم‌های انرژی}
% گروه آموزشی خود را وارد کنید
\subject{ مهندسی مکانیک }
% گرایش خود را وارد کنید
\field{مشترک مکانیک }
% عنوان پایان‌نامه را وارد کنید
\title{فرمت نگارش پایان‌نامه و رساله‌های دانشگاه صنعتی خواجه نصیرالدین طوسی}
% نام استاد(ان) راهنما را وارد کنید
\firstsupervisor{دکتر فرشاد ترابی}
\secondsupervisor{استاد راهنمای دوم  }
% نام استاد(دان) مشاور را وارد کنید. چنانچه استاد مشاور ندارید، دستور پایین را غیرفعال کنید.
%\firstadvisor{استاد مشاور اول}
%\secondadvisor{استاد مشاور دوم}
% نام استاد(دان) ممتحن را وارد کنید. چنانچه استاد مشاور ندارید، دستور پایین را غیرفعال کنید.
\firstInternal{دکتر ترابی} 
%\secondInternal{ممتحن داخلی دوم} 

%\firstExternal{ممتحن خارجی اول} 
%\secondExternal{ممتحن خارجی دوم} 
% نام نماینده تحصیلات تکمیلی را وارد کنید
%\gradmemb{نام نماینده تحصیلات تکمیلی}

%نمره به عدد. این دستور بعد از آنکه نمره داده شد اصلاح شود و در نسخه‌ی نهایی قرار بگیرد.
\grade{0.0}
%نمره به حروف. این دستور بعد از آنکه نمره داده شد اصلاح شود و در نسخه‌ی نهایی قرار بگیرد.
\gradeHarfi{صفر تمام}
%درجه در صورتی که رساله‌ی دکتری باشد. این دستور بعد از آنکه نمره داده شد اصلاح شود و در نسخه‌ی نهایی قرار بگیرد.
\rank{بد}

% نام پژوهشگر(ان) را وارد کنید
\name{فرشاد}
% \secondname{فرشاد}
% نام خانوادگی پژوهشگر(ان) را وارد کنید
\surname{ترابی}
% \secondsurname{ترابی}
%شماره دانشجویی پژوهشگر(ان) را وارد کنید
\id{8813273}
% \secondid{8810656}
% تاریخ پایان‌نامه را وارد کنید
\thesisdate{تیرماه 1396}
% کلمات کلیدی پایان‌نامه را وارد کنید
\keywords{کلمات کلیدی به تفکیک با کاما اینجا وارد شود.}
% چکیده پایان‌نامه را وارد کنید
\faabstract{
چکیده‌ی فارسی در اینجا نوشته می‌شود. دقت شود که از لحاظ دستور نگارشی، چکیده فقط و فقط باید یک پاراگراف باشد و صرفاً به کارهایی که دستاورد همین کار است بپردازد.

}
\vtitle


%%%%%%%%%%
% چند خط زیر را هرگز دستکاری نکنید.
 \thispagestyle{empty} \newpage
\Referee \thispagestyle{empty} \newpage
\Oath  \thispagestyle{empty} \newpage
\CopyRight  \thispagestyle{empty} \newpage
%%%%%%%%%%

% چنانچه مایل به چاپ صفحات «تقدیم»، «نیایش» و «سپاس‌گزاری» در خروجی نیستید، خط‌های زیر را با گذاشتن ٪  در ابتدای آنها غیرفعال کنید.
 % پایان‌نامه خود را تقدیم کنید!
% تقدیم
\begin{acknowledgementpage}

\vspace{4cm}

{\nastaliq	
تقدیم
}
\\[2cm]

این مجموعه به دوستداران علم تقدیم می‌گردد.




\end{acknowledgementpage}
         \thispagestyle{empty} \newpage
% نیایش
\baselineskip=.750cm
\thispagestyle{empty}
 


~\vspace{3.5cm}~\\

{\nastaliq
\hspace{1cm}
اگر تنها‌ترین تنها شوم، باز خدا هست
\vspace{.8cm}

\hspace{4.7cm}
او جانشین همه نداشتن‌هاست...
}
      \thispagestyle{empty} \newpage
% تشکر و قدردانی
\newpage\thispagestyle{empty}
~\vspace{4cm}

% سپاس‌گزاری
{\nastaliq
سپاس‌گزاری...
}
\\[2cm]
خداوند را شاکرم که به من چنین لطفی عنایت فرمود ...

\signature 
\newpage\clearpage  \thispagestyle{empty} \newpage

 \renewcommand{\baselinestretch}{1.3}\selectfont %\settextfont[Scale=1.2]{HM XNiloofar}
 \chapter*{چکیده}\noindent\faAbstract\par\vspace{5mm}% 
\thispagestyle{empty}\noindent{\textbf{کلمه‌های کلیدی:\ }}\Keywords
 \clearpage

%%%%%%%%%%%%%%%%%%%%%%%%%%%%%%%%%%%%

%%%%%%%%%%%%%%%%%%%%%%%%%%%%%%%%%%%%
