

 \فصل{مقدمه}


سبک متن حاضر به منظور تنظیم و نگارش پایان‌نامه‌های 
\ف{دانشگاه صنعتی خواجه نصیرالدین طوسی}دانشگاه صنعتی خواجه نصیرالدین طوسی
تهیه شده‌است.  این سبک بر مبنای سبک متن دانشگاه تبریز\فا{سبک متن دانشگاه تبریز}
 نگارش آقای وحید دامن‌افشان\ف{وحید دامن‌افشان} تهیه شده است.


از جمله مشکلات موجود در نگارش پایان‌نامه‌های دانشجویی آن است که دانشجویان
از یک الگوی یکسان جهت نگارش استفاده نمی‌کنند و زمان زیادی صرف آموزش این
الگو به آنها می‌شود. 
از خوبی‌های استفاده از لیتک آن است که دیگر نیازی به تنظیم مشخصات قسمت‌های 
مختلف و صفحه‌آرایی توسط دانشجو نبوده و تمامی تنظیمات توسط سبک حاضر اعمال می‌شود.
همچنین ترتیب قرار گرفتن صفحات اولیه‌ی پایان‌نامه و متن آنها در سبک حاضر آماده شده
و دیگر نیازی به یادآوری آن نیست.









\قسمت{\فا{پیش‌نیازهای پردازش}}
برای پردازش لازم است یک توزیع مناسب از \lr{TeX} روی سیستم نصب باشد. 
بهترین گزینه در حال حاضر توزیع \lr{texlive 2013}  است که با سبک حاضر کاملاً سازگار است.
توزیع‌های دیگر \lr{TeX} نظیر \lr{MikTeX}  توصیه نمی‌شود.

به غیر از توزیع تِک مناسب، باید فونت‌های سری \lr{HM\_X} و همچنین فونت \lr{IranNastaliq}
روی سیستم نصب باشد.
فونت‌های سری \lr{HM\_X} را می‌توان از طریق آدرس زیر دریافت نمود:
\begin{LTR}
\lr{https://bitbucket.org/dma8hm1334/persian-hm-xbs}
\end{LTR}

همچنین فونت \lr{IranNastaliq} از طریق آدرس زیر قابل دریافت است:
\begin{LTR}
\lr{http://www.parsilatex.com/joomla/index.php/remository/Fonts/}
\end{LTR}


\قسمت{\فا{از کجا شروع کنیم}}
فایل اصلی برای پردازش، \lr{Thesis.tex } است و باید پردازش روی این فایل انجام شود.
اما برای شروع به کار، ابتدا باید مشخصات دانشجو، استاد راهنما و پایان‌نامه
\ف{مشخصات!دانشجو}\ف{مشخصات!استاد راهنما}\ف{مشخصات!پایان‌نامه}
به صورت فارسی و انگلیسی وارد شود.
برای این کار کافی است فایل \lr{fa\_title.tex } که حاوی اطلاعات مربوط به 
پایان‌نامه است کامل شود. اطلاعات انگلیسی نیز در فایلی به نام    \lr{en\_title.tex }
قرار دارد.


\قسمت{\فا{آموزش مقدماتی لیتک}}
برای آموزش لیتک به طور مقدماتی فیلم‌هایی ساخته شده که 
دیدن آنها بسیار مفیداست. این فیلم‌ها را می‌توان از آدرس زیر تهیه نمود:
\begin{LTR}
\lr{http://wp.kntu.ac.ir/ftorabi/courses.html}
\end{LTR}
همچنین «\فا{تالار گفتگوی پارسی}» به منظور پشتیبانی  از بسته‌ی زی‌پرشین به آدرس زیر موجود
است که با عضویت در آن می‌توان  سوال‌های خود را مطرح کرده تا کارشناسان متفاوت
به حل آن بپردازند:

\begin{LTR}
\lr{http://www.parsilatex.com/forum}
\end{LTR}







